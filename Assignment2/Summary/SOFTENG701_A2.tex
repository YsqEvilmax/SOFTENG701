\newcommand{\Item}[2]{
	\textbf{#1:}
	\textnormal{#2}
}

\newcommand{\ImageBlock}[2]{
	\begin{figure}[ht]
		\centering
		\includegraphics[width=\columnwidth]{#1}
		\caption{#2}	
		\label{fig:#1}
	\end{figure}
}
%% bare_jrnl.tex
%% V1.4b
%% 2015/08/26
%% by Michael Shell
%% see http://www.michaelshell.org/
%% for current contact information.
%%
%% This is a skeleton file demonstrating the use of IEEEtran.cls
%% (requires IEEEtran.cls version 1.8b or later) with an IEEE
%% journal paper.
%%
%% Support sites:
%% http://www.michaelshell.org/tex/ieeetran/
%% http://www.ctan.org/pkg/ieeetran
%% and
%% http://www.ieee.org/

%%*************************************************************************
%% Legal Notice:
%% This code is offered as-is without any warranty either expressed or
%% implied; without even the implied warranty of MERCHANTABILITY or
%% FITNESS FOR A PARTICULAR PURPOSE! 
%% User assumes all risk.
%% In no event shall the IEEE or any contributor to this code be liable for
%% any damages or losses, including, but not limited to, incidental,
%% consequential, or any other damages, resulting from the use or misuse
%% of any information contained here.
%%
%% All comments are the opinions of their respective authors and are not
%% necessarily endorsed by the IEEE.
%%
%% This work is distributed under the LaTeX Project Public License (LPPL)
%% ( http://www.latex-project.org/ ) version 1.3, and may be freely used,
%% distributed and modified. A copy of the LPPL, version 1.3, is included
%% in the base LaTeX documentation of all distributions of LaTeX released
%% 2003/12/01 or later.
%% Retain all contribution notices and credits.
%% ** Modified files should be clearly indicated as such, including  **
%% ** renaming them and changing author support contact information. **
%%*************************************************************************


% *** Authors should verify (and, if needed, correct) their LaTeX system  ***
% *** with the testflow diagnostic prior to trusting their LaTeX platform ***
% *** with production work. The IEEE's font choices and paper sizes can   ***
% *** trigger bugs that do not appear when using other class files.       ***                          ***
% The testflow support page is at:
% http://www.michaelshell.org/tex/testflow/



\documentclass[journal]{IEEEtran}
%
% If IEEEtran.cls has not been installed into the LaTeX system files,
% manually specify the path to it like:
% \documentclass[journal]{../sty/IEEEtran}





% Some very useful LaTeX packages include:
% (uncomment the ones you want to load)


% *** MISC UTILITY PACKAGES ***
%
%\usepackage{ifpdf}
% Heiko Oberdiek's ifpdf.sty is very useful if you need conditional
% compilation based on whether the output is pdf or dvi.
% usage:
% \ifpdf
%   % pdf code
% \else
%   % dvi code
% \fi
% The latest version of ifpdf.sty can be obtained from:
% http://www.ctan.org/pkg/ifpdf
% Also, note that IEEEtran.cls V1.7 and later provides a builtin
% \ifCLASSINFOpdf conditional that works the same way.
% When switching from latex to pdflatex and vice-versa, the compiler may
% have to be run twice to clear warning/error messages.
\usepackage{float}
\usepackage[top=1in, bottom=1in, left=0.75in, right=0.75in]{geometry}
\usepackage{scrextend}
\addtokomafont{labelinglabel}{\sffamily}
\usepackage{graphicx}
\graphicspath{ {image/} }
\DeclareGraphicsExtensions{.pdf,.png,.jpg}





% *** CITATION PACKAGES ***
%
%\usepackage{cite}
% cite.sty was written by Donald Arseneau
% V1.6 and later of IEEEtran pre-defines the format of the cite.sty package
% \cite{} output to follow that of the IEEE. Loading the cite package will
% result in citation numbers being automatically sorted and properly
% "compressed/ranged". e.g., [1], [9], [2], [7], [5], [6] without using
% cite.sty will become [1], [2], [5]--[7], [9] using cite.sty. cite.sty's
% \cite will automatically add leading space, if needed. Use cite.sty's
% noadjust option (cite.sty V3.8 and later) if you want to turn this off
% such as if a citation ever needs to be enclosed in parenthesis.
% cite.sty is already installed on most LaTeX systems. Be sure and use
% version 5.0 (2009-03-20) and later if using hyperref.sty.
% The latest version can be obtained at:
% http://www.ctan.org/pkg/cite
% The documentation is contained in the cite.sty file itself.






% *** GRAPHICS RELATED PACKAGES ***
%
\ifCLASSINFOpdf
% \usepackage[pdftex]{graphicx}
% declare the path(s) where your graphic files are
% \graphicspath{{../pdf/}{../jpeg/}}
% and their extensions so you won't have to specify these with
% every instance of \includegraphics
% \DeclareGraphicsExtensions{.pdf,.jpeg,.png}
\else
% or other class option (dvipsone, dvipdf, if not using dvips). graphicx
% will default to the driver specified in the system graphics.cfg if no
% driver is specified.
% \usepackage[dvips]{graphicx}
% declare the path(s) where your graphic files are
% \graphicspath{{../eps/}}
% and their extensions so you won't have to specify these with
% every instance of \includegraphics
% \DeclareGraphicsExtensions{.eps}
\fi
% graphicx was written by David Carlisle and Sebastian Rahtz. It is
% required if you want graphics, photos, etc. graphicx.sty is already
% installed on most LaTeX systems. The latest version and documentation
% can be obtained at: 
% http://www.ctan.org/pkg/graphicx
% Another good source of documentation is "Using Imported Graphics in
% LaTeX2e" by Keith Reckdahl which can be found at:
% http://www.ctan.org/pkg/epslatex
%
% latex, and pdflatex in dvi mode, support graphics in encapsulated
% postscript (.eps) format. pdflatex in pdf mode supports graphics
% in .pdf, .jpeg, .png and .mps (metapost) formats. Users should ensure
% that all non-photo figures use a vector format (.eps, .pdf, .mps) and
% not a bitmapped formats (.jpeg, .png). The IEEE frowns on bitmapped formats
% which can result in "jaggedy"/blurry rendering of lines and letters as
% well as large increases in file sizes.
%
% You can find documentation about the pdfTeX application at:
% http://www.tug.org/applications/pdftex





% *** MATH PACKAGES ***
%
%\usepackage{amsmath}
% A popular package from the American Mathematical Society that provides
% many useful and powerful commands for dealing with mathematics.
%
% Note that the amsmath package sets \interdisplaylinepenalty to 10000
% thus preventing page breaks from occurring within multiline equations. Use:
%\interdisplaylinepenalty=2500
% after loading amsmath to restore such page breaks as IEEEtran.cls normally
% does. amsmath.sty is already installed on most LaTeX systems. The latest
% version and documentation can be obtained at:
% http://www.ctan.org/pkg/amsmath





% *** SPECIALIZED LIST PACKAGES ***
%
%\usepackage{algorithmic}
% algorithmic.sty was written by Peter Williams and Rogerio Brito.
% This package provides an algorithmic environment fo describing algorithms.
% You can use the algorithmic environment in-text or within a figure
% environment to provide for a floating algorithm. Do NOT use the algorithm
% floating environment provided by algorithm.sty (by the same authors) or
% algorithm2e.sty (by Christophe Fiorio) as the IEEE does not use dedicated
% algorithm float types and packages that provide these will not provide
% correct IEEE style captions. The latest version and documentation of
% algorithmic.sty can be obtained at:
% http://www.ctan.org/pkg/algorithms
% Also of interest may be the (relatively newer and more customizable)
% algorithmicx.sty package by Szasz Janos:
% http://www.ctan.org/pkg/algorithmicx




% *** ALIGNMENT PACKAGES ***
%
%\usepackage{array}
% Frank Mittelbach's and David Carlisle's array.sty patches and improves
% the standard LaTeX2e array and tabular environments to provide better
% appearance and additional user controls. As the default LaTeX2e table
% generation code is lacking to the point of almost being broken with
% respect to the quality of the end results, all users are strongly
% advised to use an enhanced (at the very least that provided by array.sty)
% set of table tools. array.sty is already installed on most systems. The
% latest version and documentation can be obtained at:
% http://www.ctan.org/pkg/array


% IEEEtran contains the IEEEeqnarray family of commands that can be used to
% generate multiline equations as well as matrices, tables, etc., of high
% quality.




% *** SUBFIGURE PACKAGES ***
%\ifCLASSOPTIONcompsoc
%  \usepackage[caption=false,font=normalsize,labelfont=sf,textfont=sf]{subfig}
%\else
%  \usepackage[caption=false,font=footnotesize]{subfig}
%\fi
% subfig.sty, written by Steven Douglas Cochran, is the modern replacement
% for subfigure.sty, the latter of which is no longer maintained and is
% incompatible with some LaTeX packages including fixltx2e. However,
% subfig.sty requires and automatically loads Axel Sommerfeldt's caption.sty
% which will override IEEEtran.cls' handling of captions and this will result
% in non-IEEE style figure/table captions. To prevent this problem, be sure
% and invoke subfig.sty's "caption=false" package option (available since
% subfig.sty version 1.3, 2005/06/28) as this is will preserve IEEEtran.cls
% handling of captions.
% Note that the Computer Society format requires a larger sans serif font
% than the serif footnote size font used in traditional IEEE formatting
% and thus the need to invoke different subfig.sty package options depending
% on whether compsoc mode has been enabled.
%
% The latest version and documentation of subfig.sty can be obtained at:
% http://www.ctan.org/pkg/subfig




% *** FLOAT PACKAGES ***
%
%\usepackage{fixltx2e}
% fixltx2e, the successor to the earlier fix2col.sty, was written by
% Frank Mittelbach and David Carlisle. This package corrects a few problems
% in the LaTeX2e kernel, the most notable of which is that in current
% LaTeX2e releases, the ordering of single and double column floats is not
% guaranteed to be preserved. Thus, an unpatched LaTeX2e can allow a
% single column figure to be placed prior to an earlier double column
% figure.
% Be aware that LaTeX2e kernels dated 2015 and later have fixltx2e.sty's
% corrections already built into the system in which case a warning will
% be issued if an attempt is made to load fixltx2e.sty as it is no longer
% needed.
% The latest version and documentation can be found at:
% http://www.ctan.org/pkg/fixltx2e


%\usepackage{stfloats}
% stfloats.sty was written by Sigitas Tolusis. This package gives LaTeX2e
% the ability to do double column floats at the bottom of the page as well
% as the top. (e.g., "\begin{figure*}[!b]" is not normally possible in
% LaTeX2e). It also provides a command:
%\fnbelowfloat
% to enable the placement of footnotes below bottom floats (the standard
% LaTeX2e kernel puts them above bottom floats). This is an invasive package
% which rewrites many portions of the LaTeX2e float routines. It may not work
% with other packages that modify the LaTeX2e float routines. The latest
% version and documentation can be obtained at:
% http://www.ctan.org/pkg/stfloats
% Do not use the stfloats baselinefloat ability as the IEEE does not allow
% \baselineskip to stretch. Authors submitting work to the IEEE should note
% that the IEEE rarely uses double column equations and that authors should try
% to avoid such use. Do not be tempted to use the cuted.sty or midfloat.sty
% packages (also by Sigitas Tolusis) as the IEEE does not format its papers in
% such ways.
% Do not attempt to use stfloats with fixltx2e as they are incompatible.
% Instead, use Morten Hogholm'a dblfloatfix which combines the features
% of both fixltx2e and stfloats:
%
% \usepackage{dblfloatfix}
% The latest version can be found at:
% http://www.ctan.org/pkg/dblfloatfix




%\ifCLASSOPTIONcaptionsoff
%  \usepackage[nomarkers]{endfloat}
% \let\MYoriglatexcaption\caption
% \renewcommand{\caption}[2][\relax]{\MYoriglatexcaption[#2]{#2}}
%\fi
% endfloat.sty was written by James Darrell McCauley, Jeff Goldberg and 
% Axel Sommerfeldt. This package may be useful when used in conjunction with 
% IEEEtran.cls'  captionsoff option. Some IEEE journals/societies require that
% submissions have lists of figures/tables at the end of the paper and that
% figures/tables without any captions are placed on a page by themselves at
% the end of the document. If needed, the draftcls IEEEtran class option or
% \CLASSINPUTbaselinestretch interface can be used to increase the line
% spacing as well. Be sure and use the nomarkers option of endfloat to
% prevent endfloat from "marking" where the figures would have been placed
% in the text. The two hack lines of code above are a slight modification of
% that suggested by in the endfloat docs (section 8.4.1) to ensure that
% the full captions always appear in the list of figures/tables - even if
% the user used the short optional argument of \caption[]{}.
% IEEE papers do not typically make use of \caption[]'s optional argument,
% so this should not be an issue. A similar trick can be used to disable
% captions of packages such as subfig.sty that lack options to turn off
% the subcaptions:
% For subfig.sty:
% \let\MYorigsubfloat\subfloat
% \renewcommand{\subfloat}[2][\relax]{\MYorigsubfloat[]{#2}}
% However, the above trick will not work if both optional arguments of
% the \subfloat command are used. Furthermore, there needs to be a
% description of each subfigure *somewhere* and endfloat does not add
% subfigure captions to its list of figures. Thus, the best approach is to
% avoid the use of subfigure captions (many IEEE journals avoid them anyway)
% and instead reference/explain all the subfigures within the main caption.
% The latest version of endfloat.sty and its documentation can obtained at:
% http://www.ctan.org/pkg/endfloat
%
% The IEEEtran \ifCLASSOPTIONcaptionsoff conditional can also be used
% later in the document, say, to conditionally put the References on a 
% page by themselves.




% *** PDF, URL AND HYPERLINK PACKAGES ***
%
%\usepackage{url}
% url.sty was written by Donald Arseneau. It provides better support for
% handling and breaking URLs. url.sty is already installed on most LaTeX
% systems. The latest version and documentation can be obtained at:
% http://www.ctan.org/pkg/url
% Basically, \url{my_url_here}.




% *** Do not adjust lengths that control margins, column widths, etc. ***
% *** Do not use packages that alter fonts (such as pslatex).         ***
% There should be no need to do such things with IEEEtran.cls V1.6 and later.
% (Unless specifically asked to do so by the journal or conference you plan
% to submit to, of course. )


% correct bad hyphenation here
\hyphenation{op-tical net-works semi-conduc-tor}


\begin{document}
	%
	% paper title
	% Titles are generally capitalized except for words such as a, an, and, as,
	% at, but, by, for, in, nor, of, on, or, the, to and up, which are usually
	% not capitalized unless they are the first or last word of the title.
	% Linebreaks \\ can be used within to get better formatting as desired.
	% Do not put math or special symbols in the title.
	\title{Add The Map Literal Feature to Java Compilers by JavaCC}
	%
	%
	% author names and IEEE memberships
	% note positions of commas and nonbreaking spaces ( ~ ) LaTeX will not break
	% a structure at a ~ so this keeps an author's name from being broken across
	% two lines.
	% use \thanks{} to gain access to the first footnote area
	% a separate \thanks must be used for each paragraph as LaTeX2e's \thanks
	% was not built to handle multiple paragraphs
	%
	
	\author{Shaoqing Yu ,~\IEEEmembership{syu702}}% <-this % stops a space
	
	% note the % following the last \IEEEmembership and also \thanks - 
	% these prevent an unwanted space from occurring between the last author name
	% and the end of the author line. i.e., if you had this:
	% 
	% \author{....lastname \thanks{...} \thanks{...} }
	%                     ^------------^------------^----Do not want these spaces!
	%
	% a space would be appended to the last name and could cause every name on that
	% line to be shifted left slightly. This is one of those "LaTeX things". For
	% instance, "\textbf{A} \textbf{B}" will typeset as "A B" not "AB". To get
	% "AB" then you have to do: "\textbf{A}\textbf{B}"
	% \thanks is no different in this regard, so shield the last } of each \thanks
	% that ends a line with a % and do not let a space in before the next \thanks.
	% Spaces after \IEEEmembership other than the last one are OK (and needed) as
	% you are supposed to have spaces between the names. For what it is worth,
	% this is a minor point as most people would not even notice if the said evil
	% space somehow managed to creep in.
	
	
	
	% The paper headers
	\markboth{April~2016}{}
	% The only time the second header will appear is for the odd numbered pages
	% after the title page when using the twoside option.
	% 
	% *** Note that you probably will NOT want to include the author's ***
	% *** name in the headers of peer review papers.                   ***
	% You can use \ifCLASSOPTIONpeerreview for conditional compilation here if
	% you desire.
	
	
	
	
	% If you want to put a publisher's ID mark on the page you can do it like
	% this:
	%\IEEEpubid{0000--0000/00\$00.00~\copyright~2015 IEEE}
	% Remember, if you use this you must call \IEEEpubidadjcol in the second
	% column for its text to clear the IEEEpubid mark.
	
	
	
	% use for special paper notices
	%\IEEEspecialpapernotice{(Invited Paper)}
	
	
	
	
	% make the title area
	\maketitle
	
	% As a general rule, do not put math, special symbols or citations
	% in the abstract or keywords.
	\begin{abstract}
		The map literal feature, which is commonly used by some script languages, is on demand by more and more Java coders, who believe it can improve the code readability and coding convenience. This report is written for demonstrating an approach to add the map literal recognition ability to a Java compiler.
	\end{abstract}
	
	% Note that keywords are not normally used for peerreview papers.
	\begin{IEEEkeywords}
		Map literal, semantic analysis, symbol table, Java complier.
	\end{IEEEkeywords}
	
	
	
	
	
	
	% For peer review papers, you can put extra information on the cover
	% page as needed:
	% \ifCLASSOPTIONpeerreview
	% \begin{center} \bfseries EDICS Category: 3-BBND \end{center}
	% \fi
	%
	% For peerreview papers, this IEEEtran command inserts a page break and
	% creates the second title. It will be ignored for other modes.
	\IEEEpeerreviewmaketitle
	
	
	
	\section{Introduction}
	% The very first letter is a 2 line initial drop letter followed
	% by the rest of the first word in caps.
	% 
	% form to use if the first word consists of a single letter:
	% \IEEEPARstart{A}{demo} file is ....
	% 
	% form to use if you need the single drop letter followed by
	% normal text (unknown if ever used by the IEEE):
	% \IEEEPARstart{A}{}demo file is ....
	% 
	% Some journals put the first two words in caps:
	% \IEEEPARstart{T}{his demo} file is ....
	% 
	% Here we have the typical use of a "T" for an initial drop letter
	% and "HIS" in caps to complete the first word.
	\IEEEPARstart{J}{ava} is a popular but still on-growing programming language, which has long been integrating new features, good or not, from others. Recently, as Java has become one of the main web development language, more and more coders, especially those who wrote scripts mainly before, hope the map literal can be supported by Java compiler. This report will analyse the reasons for this demand first, and then show an approach to add this feature by JavaCC. 
	
	\hfill Shaoqing Yu
	
	\hfill April 11, 2016
	
	
	% An example of a floating figure using the graphicx package.
	% Note that \label must occur AFTER (or within) \caption.
	% For figures, \caption should occur after the \includegraphics.
	% Note that IEEEtran v1.7 and later has special internal code that
	% is designed to preserve the operation of \label within \caption
	% even when the captionsoff option is in effect. However, because
	% of issues like this, it may be the safest practice to put all your
	% \label just after \caption rather than within \caption{}.
	%
	% Reminder: the "draftcls" or "draftclsnofoot", not "draft", class
	% option should be used if it is desired that the figures are to be
	% displayed while in draft mode.
	%
	%\begin{figure}[!t]
	%\centering
	%\includegraphics[width=2.5in]{myfigure}
	% where an .eps filename suffix will be assumed under latex, 
	% and a .pdf suffix will be assumed for pdflatex; or what has been declared
	% via \DeclareGraphicsExtensions.
	%\caption{Simulation results for the network.}
	%\label{fig_sim}
	%\end{figure}
	
	% Note that the IEEE typically puts floats only at the top, even when this
	% results in a large percentage of a column being occupied by floats.
	
	
	% An example of a double column floating figure using two subfigures.
	% (The subfig.sty package must be loaded for this to work.)
	% The subfigure \label commands are set within each subfloat command,
	% and the \label for the overall figure must come after \caption.
	% \hfil is used as a separator to get equal spacing.
	% Watch out that the combined width of all the subfigures on a 
	% line do not exceed the text width or a line break will occur.
	%
	%\begin{figure*}[!t]
	%\centering
	%\subfloat[Case I]{\includegraphics[width=2.5in]{box}%
	%\label{fig_first_case}}
	%\hfil
	%\subfloat[Case II]{\includegraphics[width=2.5in]{box}%
	%\label{fig_second_case}}
	%\caption{Simulation results for the network.}
	%\label{fig_sim}
	%\end{figure*}
	%
	% Note that often IEEE papers with subfigures do not employ subfigure
	% captions (using the optional argument to \subfloat[]), but instead will
	% reference/describe all of them (a), (b), etc., within the main caption.
	% Be aware that for subfig.sty to generate the (a), (b), etc., subfigure
	% labels, the optional argument to \subfloat must be present. If a
	% subcaption is not desired, just leave its contents blank,
	% e.g., \subfloat[].
	
	
	% An example of a floating table. Note that, for IEEE style tables, the
	% \caption command should come BEFORE the table and, given that table
	% captions serve much like titles, are usually capitalized except for words
	% such as a, an, and, as, at, but, by, for, in, nor, of, on, or, the, to
	% and up, which are usually not capitalized unless they are the first or
	% last word of the caption. Table text will default to \footnotesize as
	% the IEEE normally uses this smaller font for tables.
	% The \label must come after \caption as always.
	%
	%\begin{table}[!t]
	%% increase table row spacing, adjust to taste
	%\renewcommand{\arraystretch}{1.3}
	% if using array.sty, it might be a good idea to tweak the value of
	% \extrarowheight as needed to properly center the text within the cells
	%\caption{An Example of a Table}
	%\label{table_example}
	%\centering
	%% Some packages, such as MDW tools, offer better commands for making tables
	%% than the plain LaTeX2e tabular which is used here.
	%\begin{tabular}{|c||c|}
	%\hline
	%One & Two\\
	%\hline
	%Three & Four\\
	%\hline
	%\end{tabular}
	%\end{table}
	
	
	% Note that the IEEE does not put floats in the very first column
	% - or typically anywhere on the first page for that matter. Also,
	% in-text middle ("here") positioning is typically not used, but it
	% is allowed and encouraged for Computer Society conferences (but
	% not Computer Society journals). Most IEEE journals/conferences use
	% top floats exclusively. 
	% Note that, LaTeX2e, unlike IEEE journals/conferences, places
	% footnotes above bottom floats. This can be corrected via the
	% \fnbelowfloat command of the stfloats package.
	
	
	\section{Motivation}
	
	Compared with the map initialization in some script languages, such as Perl \cite{Wall} and PHP \cite{Tatroe}, the map initialization in Java used to be viewed as a weakness as it is laborious and syntactically inefficient \cite{Buesing, Minborg}. The latest release, Java 8, tries to improve the map in-line initialization from JDK level by some new features called double brace initializer or static initialization \cite{Arnold, www.c2.com, Minborg, Bansal}. However, it does reduce a little bit work on map initialization but probably brings more performance trade-off and even potential errors in real practices \cite{jOOQ, StackOverFlow}. If the map literal feature can be supported from the compiling level, the optimized code will be pure Java, which has neither performance penalty nor potential fatal bug. Therefore, the map literal is a demanding feature which should be implemented on compiling level. 
	
	\section{implementation}
	
	There are 2 steps in implementing the Map literal feature: the common semantic analysis and the map literal feature integration.
	
	\subsection{Semantic Analysis}
	
	The semantic analysis mainly consists of tracking class, method and variable declarations and their type checking. It can be roughly divided into 2 parts: the in-scope existence checking and the type compatibility checking. To determine the accessibility of identifiers and then analyse them in logics, a scope stack properly filled with symbol tables is required to be defined and generated first.
	
	\subsubsection{Symbol Table}
	
	\ImageBlock{symtab_arc}{The UML class diagram of the symbol table.}
	
	The scope stack need to be properly organised by symbol tables filled with different symbols at each scope point. The classes required in the scope stack creation are showed in Figure~\ref{fig:symtab_arc}. When a node associated with built in type, variable, enum, method, interface or class is visited during AST traversing, a symbol need to be created and loaded to the current symbol table. Sometimes if the symbol itself is a scope, for example the ClassSymbol, it will be pushed in to scope stack as a scope and be expanded at this point. 
	
	\subsubsection{Visitors}
	
	\ImageBlock{visitors_call}{The work flow diagram of visitors newly added.}
	
	The actual operations on symbols in current scope are performed by visitors, which inject a particular action to each node in AST. The visitors will be executed in the sequence showed in Figure~\ref{fig:visitors_call}. Accordingly to the 2 main parts of semantic analysis we mentioned before, they can be categorized into 2 groups to, firstly, create the scope stack (ScopeCreateVisitor) and define symbols (ClassCollectVisitor, MethodCollectVisitor, VariableCollectVisitot) when traversing the AST, and then check all the symbols in scope stack logically from different perspectives (class, method, variable and map).
	
	\ImageBlock{visitors_arc}{The UML class diagram of visitors newly added.}
	
	The visitor class inheritance is designed as the Figure~\ref{fig:visitors_arc} demonstrates. To improve the re-usability of code, an abstract class named TravelVisitor is created to provide the AST traversing ability to all its child classes, unless they override their own visit functions. Each visitor inheriting from TypeCheckVisitor shares the basic type checking function to check the type compatibility of identifiers, while, applying it in different points (class, method, variable, map). The detailed descriptions of each visitor are listed below:
	
	\begin{labeling}{alligator}
		\item [\textbf{TravelVisitor}] {It traverses through the AST by visiting all the nodes.}
		\item [\textbf{ScopeCreateVisitor}] {Based on the traversing ability from TravelVisitor, it initializes the current scope of each node in AST, meanwhile creating and expanding a new scope if necessary.}
		\item [\textbf{ClassCollectVisitor}] {Based on the traversing ability from TravelVisitor, it defines every class symbol and puts them to the proper scope when visiting ClassOrInterfaceDeclaration node in the AST.}
		\item [\textbf{MethodCollectVisitor}] {Based on the traversing ability from TravelVisitor, it defines every method symbol and puts them to the proper scope when ConstructorDeclaration and MethodDeclaration are visited in the AST.}
		\item [\textbf{VariableCollectVisitor}] {Based on the traversing ability from TravelVisitor, it defines every variable symbol and puts them to the proper scope when VariableDeclaratorId is caught in the AST.}
		\item [\textbf{TypeCheckVisitor}] {Based on the traversing ability from TravelVisitor, it checks if the type of the target symbol matches the expecting type.}
		\item [\textbf{ClassCheckVisitor}] {With the type checking ability from TypeCheckVisitor, it checks all the members (methods, fields and inheritance) in a class when traversing the AST.}
		\item [\textbf{MethodCheckVisitor}] {With the type checking ability from TypeCheckVisitor, it checks return statements, constructors and methods when traversing the AST.}
		\item [\textbf{VariableCheckVisitor}] {With the type checking ability from TypeCheckVisitor, it checks all the variable names in declarations and assign statements when traversing the AST.}
		\item [\textbf{MapCheckVisitor}] {With the type checking ability from TypeCheckVisitor, it checks the size and types consistency of key-value pairs in the Map nodes when traversing the AST.}
	\end{labeling}
	
	\subsection{Map Literal checking}
	
	After the common semantic analysis has been achieved, the next is to add the new map literal feature.
	
	\subsubsection{Lexicon Level}
	
	\ImageBlock{map_item}{The detailed code of MapItem.}
	
	At this level, firstly, a new production named MapItem, which is able to catch a literal or variable name and then return the expression, should be created like Figure~\ref{fig:map_item}.
	
	\ImageBlock{map_create}{The detailed code of modification about AllocationExpression.}
	
	Then, the AllocationExpression should be modified by adding a new option to parse out the map literal when the tokens associated with map literals are caught after "new" key word. See main code in Figure~\ref{fig:map_create}.
	
	\subsubsection{Syntactic Level}
	
	\ImageBlock{map_expr}{The definition of MapInitializationExpr.}
	
	Once the map literal is caught, a new AST node named MapInitializationExpr will be created and added to the AST. We can see from Figure~\ref{fig:map_expr} that it wraps a list of key and value expressions respectively, as well as its own expression type. 
	
	\subsubsection{Semantic Level}
	
	As we claimed before, the type checking of HashMap is defined by TypeCheckVisitor and executed by VariableCheckVisitor as it inherits from TypeCheckVisitor and checks the variable type, no matter the variable is HashMap or something else. Even though, specific to the MapCheckVisitor, we still need to do some checking about the Map itself. This includes the size equality of values and keys, and the type consistency among either keys or values.
	
	Here is still one more trick we need to play. The primitive type name can not be used directly in Map declaration, for example, the "Integer" should be used instead of "int" in the type argument of Map. We also need to convert the name of primitive type used in Map during type checking.
	
	\subsubsection{Display}
	
	In DumpVisitor, if MapInitializationExpr is recognised, the display way, including the printing of type, variable name and map literal, will be all delayed until calling the funtion "visit(MapInitializationExpr , Object)" according to the original Java Grammar. The types of keys and values, no matter are explicitly declared before or implicitly detected here and filled in VarArgs, will be displayed as the map variable arguments in a pair of "<>" after "HashMap". Then traverse through kay-value pairs to print Map.put(key, value) after the HashMap declaration.
	
	\subsubsection{Acheivement}
	Those points below are soundly tested by 13 cases in "extentiontests" folder.
		\begin{labeling}{alligator}
			\item [\textbf{Map literal translation}] {Map literals can be translated into pure Java code.}
			\item [\textbf{Multiple types}] {Not ony primitive type, but also customized type can be used in map literals.}
			\item [\textbf{Division of map declaration and assignment}] {Map can be declared as a member of a class and then assigned by a map literal.}
			\item [\textbf{Multiple map initalization inline}] {Many variables of maps can be initialized by map literals inline once.}
		\end{labeling}
	 
	\section{Restriction}
	\Item{No in Class Members}{The map literal cannot be used as a class member declaration, because the compiler have no idea about where to put the callings of Map.put().}
	
	\Item{No in Return Statements}{The map literal cannot be used in return expression directly without assigned by a variable.}
	
	\Item{Map \& HashMap only}{In this implementation, the map literal can only be used between Map and HashMap.}
	
	\section{Conclusion}
	In this report, we have discussed the advantages of the Map literal, then described an approach to implement this new feature, as well as the partial semantic analysis in Java code. It is noted that the semantic analysis is a larger topic than what we discussed here and the real cases could be much more complicated especially if some object-orientation features are concerned. The map literal is a small but very useful feature which can reduce the repeated syntactic and improve code readability. By keeping integrating new features like this, Java will be more widely used as  barriers of turning to Java from other script languages are being eliminating.
	
	
	
	% use section* for acknowledgment
	\section*{Acknowledgement}
	
	I would like to express my special thanks of gratitude to my lecturer Kelly, who is always patient to my stupid questions about JavaCC. If there exists any marker on this assignment, I think he/she definitely also deserves my thanks, because reading and testing such a massive code must be torturing and time-consuming.
	
	% Can use something like this to put references on a page
	% by themselves when using endfloat and the captionsoff option.
	\ifCLASSOPTIONcaptionsoff
	\newpage
	\fi
	
	
	
	% trigger a \newpage just before the given reference
	% number - used to balance the columns on the last page
	% adjust value as needed - may need to be readjusted if
	% the document is modified later
	%\IEEEtriggeratref{8}
	% The "triggered" command can be changed if desired:
	%\IEEEtriggercmd{\enlargethispage{-5in}}
	
	% references section
	
	% can use a bibliography generated by BibTeX as a .bbl file
	% BibTeX documentation can be easily obtained at:
	% http://mirror.ctan.org/biblio/bibtex/contrib/doc/
	% The IEEEtran BibTeX style support page is at:
	% http://www.michaelshell.org/tex/ieeetran/bibtex/
	%\bibliographystyle{IEEEtran}
	% argument is your BibTeX string definitions and bibliography database(s)
	%\bibliography{IEEEabrv,../bib/paper}
	%
	% <OR> manually copy in the resultant .bbl file
	% set second argument of \begin to the number of references
	% (used to reserve space for the reference number labels box)
	\begin{thebibliography}{9}

		\bibitem{Tatroe}
		K. Tatroe, P. MacIntyre and R. Lerdorf, "Arrays", in~\emph{Programming PHP}, 7th ed. USA: O'Reilly Media, Inc., 2013, ch. 2, sec. 2, p. 26,27. 
			
		\bibitem{Arnold}
		K. Arnold, J. Gosling and D. Holmes, "Static Initialization", in~\emph{THE Java™ Programming Language}, 4th ed. Boston, USA: Addison Wesley Professional, 2005, ch. 2, sec. 5, p. 75. 				
		
		\bibitem{Wall}
		L. Wall, T. Christiansen and J. Orwant, "Typing Hashes", in~\emph{Programming Perl}, 3rd ed. USA: O'Reilly Media, Inc., 2000, ch. 14, sec. 3, pp. 378-384. 
			
		\bibitem{www.c2.com}
		www.c2.com, \emph{Double Brace Initialization}, 2014, [online]. Available: http://www.c2.com/cgi/wiki?DoubleBraceInitialization. [Accessed: 11- Apr- 2016].
		
		\bibitem{jOOQ}
		jOOQ, \emph{Don’t be “Clever”: The Double Curly Braces Anti Pattern}, 2014, [online]. Available: https://blog.jooq.org/2014/12/08/dont-be-clever-the-double-curly-braces-anti-pattern/. [Accessed: 11- Apr- 2016].		
		
		\bibitem{Buesing}
		N.~Buesing, \emph{Inline initialization of Java Maps}, 2014, [online]. Available: https://objectpartners.com/2014/06/05/inline-initialization-of-java-maps/. [Accessed: 10- Apr- 2016].		
		
		\bibitem{Minborg}
		P.~A.~Minborg, \emph{Java 8, Initializing Maps in the Smartest Way}, 2014, [online]. Available: http://minborgsjavapot.blogspot.co.nz/2014/12/java-8-initializing-maps-in-smartest-way.html. [Accessed: 10- Apr- 2016].
		
		\bibitem{Bansal}
		N. Bansal, \emph{Initializing Java Maps Inline}, 2009, [online]. Available: http://nileshbansal.blogspot.co.nz/2009/04/initializing-java-maps-inline.html. [Accessed: 11- Apr- 2016].		
	
		\bibitem{StackOverFlow}
		StackOverFlow, \emph{Efficiency of Java “Double Brace Initialization”?}, 2009, [online]. Available: https://stackoverflow.com/questions/924285/efficiency-of-java-double-brace-initialization. [Accessed: 11- Apr- 2016].
			
	\end{thebibliography}
	
	% You can push biographies down or up by placing
	% a \vfill before or after them. The appropriate
	% use of \vfill depends on what kind of text is
	% on the last page and whether or not the columns
	% are being equalized.
	
	%\vfill
	
	% Can be used to pull up biographies so that the bottom of the last one
	% is flush with the other column.
	%\enlargethispage{-5in}
	
	
	
	% that's all folks
\end{document}


