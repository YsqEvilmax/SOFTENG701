\newcommand{\Item}[2]{
	\textbf{#1:}
	\textnormal{#2}
}

%% bare_conf.tex
%% V1.4b
%% 2015/08/26
%% by Michael Shell
%% See:
%% http://www.michaelshell.org/
%% for current contact information.
%%
%% This is a skeleton file demonstrating the use of IEEEtran.cls
%% (requires IEEEtran.cls version 1.8b or later) with an IEEE
%% conference paper.
%%
%% Support sites:
%% http://www.michaelshell.org/tex/ieeetran/
%% http://www.ctan.org/pkg/ieeetran
%% and
%% http://www.ieee.org/

%%*************************************************************************
%% Legal Notice:
%% This code is offered as-is without any warranty either expressed or
%% implied; without even the implied warranty of MERCHANTABILITY or
%% FITNESS FOR A PARTICULAR PURPOSE! 
%% User assumes all risk.
%% In no event shall the IEEE or any contributor to this code be liable for
%% any damages or losses, including, but not limited to, incidental,
%% consequential, or any other damages, resulting from the use or misuse
%% of any information contained here.
%%
%% All comments are the opinions of their respective authors and are not
%% necessarily endorsed by the IEEE.
%%
%% This work is distributed under the LaTeX Project Public License (LPPL)
%% ( http://www.latex-project.org/ ) version 1.3, and may be freely used,
%% distributed and modified. A copy of the LPPL, version 1.3, is included
%% in the base LaTeX documentation of all distributions of LaTeX released
%% 2003/12/01 or later.
%% Retain all contribution notices and credits.
%% ** Modified files should be clearly indicated as such, including  **
%% ** renaming them and changing author support contact information. **
%%*************************************************************************


% *** Authors should verify (and, if needed, correct) their LaTeX system  ***
% *** with the testflow diagnostic prior to trusting their LaTeX platform ***
% *** with production work. The IEEE's font choices and paper sizes can   ***
% *** trigger bugs that do not appear when using other class files.       ***                          ***
% The testflow support page is at:
% http://www.michaelshell.org/tex/testflow/



\documentclass[conference]{IEEEtran}
% Some Computer Society conferences also require the compsoc mode option,
% but others use the standard conference format.
%
% If IEEEtran.cls has not been installed into the LaTeX system files,
% manually specify the path to it like:
% \documentclass[conference]{../sty/IEEEtran}





% Some very useful LaTeX packages include:
% (uncomment the ones you want to load)


% *** MISC UTILITY PACKAGES ***
%
%\usepackage{ifpdf}
% Heiko Oberdiek's ifpdf.sty is very useful if you need conditional
% compilation based on whether the output is pdf or dvi.
% usage:
% \ifpdf
%   % pdf code
% \else
%   % dvi code
% \fi
% The latest version of ifpdf.sty can be obtained from:
% http://www.ctan.org/pkg/ifpdf
% Also, note that IEEEtran.cls V1.7 and later provides a builtin
% \ifCLASSINFOpdf conditional that works the same way.
% When switching from latex to pdflatex and vice-versa, the compiler may
% have to be run twice to clear warning/error messages.
\usepackage{float}
\usepackage[top=1in, bottom=1in, left=0.75in, right=0.75in]{geometry}
\usepackage{scrextend}
\addtokomafont{labelinglabel}{\sffamily}

\usepackage{amssymb}

\usepackage{caption}
\usepackage{graphicx}
\graphicspath{ {image/} }
\DeclareGraphicsExtensions{.pdf,.png,.jpg}

\newcommand{\ImageBlock}[2]{
	\begin{figure}[ht]
		\centering
		\includegraphics[width=\columnwidth]{#1}
		\caption{#2}	
		\label{fig:#1}
	\end{figure}
}


\newcommand{\ImagePage}[2]{
	\begin{minipage}[c]{\textwidth}% to keep image and caption on one page
		\centering
		\includegraphics[width=\textwidth]{#1}
		\captionof{figure}{#2}	
		\label{fig:#1}
	\end{minipage}
}

\usepackage{tabularx}  % for 'tabularx' environment and 'X' column type
\usepackage{booktabs}  % for \toprule, \midrule, and \bottomrule macros 
\usepackage{caption}
\usepackage{arydshln}

\newcommand{\tabincell}[2]{
	\begin{tabular}{@{}#1@{}}#2\end{tabular}
}

\newcommand{\TabFooter}[2]{
	\multicolumn{7}{p{\columnwidth}}{\textsuperscript{#1}\footnotesize{#2}} \\
}


\newcommand{\TableBlock}[3]{
	\begin{table}[H]
		\renewcommand{\arraystretch}{1.3}	
		\caption{#2}
		\label{tab:#1}
		\centering
		\begin{tabularx}{\columnwidth}{@{}llccccc@{}}
			#3
		\end{tabularx}
	\end{table}
}


\newcommand{\TabMetricHeader}{
	\textbf{PACKAGE} & \textbf{TYPE} & \textbf{CBO} & \textbf{DIT} & \textbf{NOC} & \textbf{WMC} & \textbf{NOSM} \\
}

\newcommand{\TabMetricFooter}{
		\TabFooter{1}{CBO: Coupling between Object Classes}
		\TabFooter{2}{DIT: Depth of Inheritance Tree}
		\TabFooter{3}{NOC: Number of Children}
		\TabFooter{4}{WMC: Weighted Methods Per Class}
		\TabFooter{5}{NOSM: Number of Static Methods}
}

\newcommand{\TabMetricColumn}[7]{
	\textbf{#1} & \textbf{#2} & #3 & #4 & #5 & #6 & #7 \\
}





% *** CITATION PACKAGES ***
%
%\usepackage{cite}
% cite.sty was written by Donald Arseneau
% V1.6 and later of IEEEtran pre-defines the format of the cite.sty package
% \cite{} output to follow that of the IEEE. Loading the cite package will
% result in citation numbers being automatically sorted and properly
% "compressed/ranged". e.g., [1], [9], [2], [7], [5], [6] without using
% cite.sty will become [1], [2], [5]--[7], [9] using cite.sty. cite.sty's
% \cite will automatically add leading space, if needed. Use cite.sty's
% noadjust option (cite.sty V3.8 and later) if you want to turn this off
% such as if a citation ever needs to be enclosed in parenthesis.
% cite.sty is already installed on most LaTeX systems. Be sure and use
% version 5.0 (2009-03-20) and later if using hyperref.sty.
% The latest version can be obtained at:
% http://www.ctan.org/pkg/cite
% The documentation is contained in the cite.sty file itself.






% *** GRAPHICS RELATED PACKAGES ***
%
\ifCLASSINFOpdf
% \usepackage[pdftex]{graphicx}
% declare the path(s) where your graphic files are
% \graphicspath{{../pdf/}{../jpeg/}}
% and their extensions so you won't have to specify these with
% every instance of \includegraphics
% \DeclareGraphicsExtensions{.pdf,.jpeg,.png}
\else
% or other class option (dvipsone, dvipdf, if not using dvips). graphicx
% will default to the driver specified in the system graphics.cfg if no
% driver is specified.
% \usepackage[dvips]{graphicx}
% declare the path(s) where your graphic files are
% \graphicspath{{../eps/}}
% and their extensions so you won't have to specify these with
% every instance of \includegraphics
% \DeclareGraphicsExtensions{.eps}
\fi
% graphicx was written by David Carlisle and Sebastian Rahtz. It is
% required if you want graphics, photos, etc. graphicx.sty is already
% installed on most LaTeX systems. The latest version and documentation
% can be obtained at: 
% http://www.ctan.org/pkg/graphicx
% Another good source of documentation is "Using Imported Graphics in
% LaTeX2e" by Keith Reckdahl which can be found at:
% http://www.ctan.org/pkg/epslatex
%
% latex, and pdflatex in dvi mode, support graphics in encapsulated
% postscript (.eps) format. pdflatex in pdf mode supports graphics
% in .pdf, .jpeg, .png and .mps (metapost) formats. Users should ensure
% that all non-photo figures use a vector format (.eps, .pdf, .mps) and
% not a bitmapped formats (.jpeg, .png). The IEEE frowns on bitmapped formats
% which can result in "jaggedy"/blurry rendering of lines and letters as
% well as large increases in file sizes.
%
% You can find documentation about the pdfTeX application at:
% http://www.tug.org/applications/pdftex





% *** MATH PACKAGES ***
%
\usepackage{amsmath}
% A popular package from the American Mathematical Society that provides
% many useful and powerful commands for dealing with mathematics.
%
% Note that the amsmath package sets \interdisplaylinepenalty to 10000
% thus preventing page breaks from occurring within multiline equations. Use:
%\interdisplaylinepenalty=2500
% after loading amsmath to restore such page breaks as IEEEtran.cls normally
% does. amsmath.sty is already installed on most LaTeX systems. The latest
% version and documentation can be obtained at:
% http://www.ctan.org/pkg/amsmath





% *** SPECIALIZED LIST PACKAGES ***
%
%\usepackage{algorithmic}
% algorithmic.sty was written by Peter Williams and Rogerio Brito.
% This package provides an algorithmic environment fo describing algorithms.
% You can use the algorithmic environment in-text or within a figure
% environment to provide for a floating algorithm. Do NOT use the algorithm
% floating environment provided by algorithm.sty (by the same authors) or
% algorithm2e.sty (by Christophe Fiorio) as the IEEE does not use dedicated
% algorithm float types and packages that provide these will not provide
% correct IEEE style captions. The latest version and documentation of
% algorithmic.sty can be obtained at:
% http://www.ctan.org/pkg/algorithms
% Also of interest may be the (relatively newer and more customizable)
% algorithmicx.sty package by Szasz Janos:
% http://www.ctan.org/pkg/algorithmicx




% *** ALIGNMENT PACKAGES ***
%
%\usepackage{array}
% Frank Mittelbach's and David Carlisle's array.sty patches and improves
% the standard LaTeX2e array and tabular environments to provide better
% appearance and additional user controls. As the default LaTeX2e table
% generation code is lacking to the point of almost being broken with
% respect to the quality of the end results, all users are strongly
% advised to use an enhanced (at the very least that provided by array.sty)
% set of table tools. array.sty is already installed on most systems. The
% latest version and documentation can be obtained at:
% http://www.ctan.org/pkg/array


% IEEEtran contains the IEEEeqnarray family of commands that can be used to
% generate multiline equations as well as matrices, tables, etc., of high
% quality.




% *** SUBFIGURE PACKAGES ***
%\ifCLASSOPTIONcompsoc
%  \usepackage[caption=false,font=normalsize,labelfont=sf,textfont=sf]{subfig}
%\else
%  \usepackage[caption=false,font=footnotesize]{subfig}
%\fi
% subfig.sty, written by Steven Douglas Cochran, is the modern replacement
% for subfigure.sty, the latter of which is no longer maintained and is
% incompatible with some LaTeX packages including fixltx2e. However,
% subfig.sty requires and automatically loads Axel Sommerfeldt's caption.sty
% which will override IEEEtran.cls' handling of captions and this will result
% in non-IEEE style figure/table captions. To prevent this problem, be sure
% and invoke subfig.sty's "caption=false" package option (available since
% subfig.sty version 1.3, 2005/06/28) as this is will preserve IEEEtran.cls
% handling of captions.
% Note that the Computer Society format requires a larger sans serif font
% than the serif footnote size font used in traditional IEEE formatting
% and thus the need to invoke different subfig.sty package options depending
% on whether compsoc mode has been enabled.
%
% The latest version and documentation of subfig.sty can be obtained at:
% http://www.ctan.org/pkg/subfig




% *** FLOAT PACKAGES ***
%
%\usepackage{fixltx2e}
% fixltx2e, the successor to the earlier fix2col.sty, was written by
% Frank Mittelbach and David Carlisle. This package corrects a few problems
% in the LaTeX2e kernel, the most notable of which is that in current
% LaTeX2e releases, the ordering of single and double column floats is not
% guaranteed to be preserved. Thus, an unpatched LaTeX2e can allow a
% single column figure to be placed prior to an earlier double column
% figure.
% Be aware that LaTeX2e kernels dated 2015 and later have fixltx2e.sty's
% corrections already built into the system in which case a warning will
% be issued if an attempt is made to load fixltx2e.sty as it is no longer
% needed.
% The latest version and documentation can be found at:
% http://www.ctan.org/pkg/fixltx2e


%\usepackage{stfloats}
% stfloats.sty was written by Sigitas Tolusis. This package gives LaTeX2e
% the ability to do double column floats at the bottom of the page as well
% as the top. (e.g., "\begin{figure*}[!b]" is not normally possible in
% LaTeX2e). It also provides a command:
%\fnbelowfloat
% to enable the placement of footnotes below bottom floats (the standard
% LaTeX2e kernel puts them above bottom floats). This is an invasive package
% which rewrites many portions of the LaTeX2e float routines. It may not work
% with other packages that modify the LaTeX2e float routines. The latest
% version and documentation can be obtained at:
% http://www.ctan.org/pkg/stfloats
% Do not use the stfloats baselinefloat ability as the IEEE does not allow
% \baselineskip to stretch. Authors submitting work to the IEEE should note
% that the IEEE rarely uses double column equations and that authors should try
% to avoid such use. Do not be tempted to use the cuted.sty or midfloat.sty
% packages (also by Sigitas Tolusis) as the IEEE does not format its papers in
% such ways.
% Do not attempt to use stfloats with fixltx2e as they are incompatible.
% Instead, use Morten Hogholm'a dblfloatfix which combines the features
% of both fixltx2e and stfloats:
%
% \usepackage{dblfloatfix}
% The latest version can be found at:
% http://www.ctan.org/pkg/dblfloatfix




% *** PDF, URL AND HYPERLINK PACKAGES ***
%
%\usepackage{url}
% url.sty was written by Donald Arseneau. It provides better support for
% handling and breaking URLs. url.sty is already installed on most LaTeX
% systems. The latest version and documentation can be obtained at:
% http://www.ctan.org/pkg/url
% Basically, \url{my_url_here}.


% *** GLOSSARY PACKAGE ***
%
\usepackage[acronym]{glossaries}
% generates the acronym list
\makeglossaries
\newacronym{OO}{OO}{Object-Oriented}

\newacronym{NoP}{NoP}{Number of Packages}
\newacronym{NoC}{NoC}{Number of Classes}
\newacronym{NoEC}{NoEC}{Number of Exception Classes}
\newacronym{NoI}{NoI}{Number of Interfaces}
\newacronym{NoE}{NoE}{Number of Enums}
\newacronym{NoT}{NoT}{Number of Types}

\newacronym{CBO}{CBO}{Coupling between Object Classes}
\newacronym{DIT}{DIT}{Depth of Inheritance Tree}
\newacronym{NOC}{NOC}{NOC Number of Children}
\newacronym{WMC}{WMC}{WMC Weighted Methods Per Class}
\newacronym{NOSM}{NOSM}{Number of Static Methods}

\newglossaryentry{elite}{name={\textbf{Elite}},description={This is an elite glossary}}
\newglossaryentry{master}{name={\textbf{Master}},description={~~~~~~~This is a master glossary}}
\newglossaryentry{masterabd}{name={\textbf{Mastera}},description={~~~~~~~~~~This is a master glossary}}

% *** Do not adjust lengths that control margins, column widths, etc. ***
% *** Do not use packages that alter fonts (such as pslatex).         ***
% There should be no need to do such things with IEEEtran.cls V1.6 and later.
% (Unless specifically asked to do so by the journal or conference you plan
% to submit to, of course. )


% correct bad hyphenation here
\hyphenation{op-tical net-works semi-conduc-tor}


\begin{document}
	%
	% paper title
	% Titles are generally capitalized except for words such as a, an, and, as,
	% at, but, by, for, in, nor, of, on, or, the, to and up, which are usually
	% not capitalized unless they are the first or last word of the title.
	% Linebreaks \\ can be used within to get better formatting as desired.
	% Do not put math or special symbols in the title.
	\title{The Modifiability Evaluation of 4 Designs of the Kalah Game}
	
	
	% author names and affiliations
	% use a multiple column layout for up to three different
	% affiliations
	\author{\IEEEauthorblockN{Shaoqing Yu (syu702)}
		\IEEEauthorblockA{Student in\\Department of Electrical and\\Computer Engineering\\
			The University of Auckland\\
			Auckland, New Zealand\\
			Email: syu702@aucklanduni.ac.nz}
	}
	
	% conference papers do not typically use \thanks and this command
	% is locked out in conference mode. If really needed, such as for
	% the acknowledgment of grants, issue a \IEEEoverridecommandlockouts
	% after \documentclass
	
	% for over three affiliations, or if they all won't fit within the width
	% of the page, use this alternative format:
	% 
	%\author{\IEEEauthorblockN{Michael Shell\IEEEauthorrefmark{1},
	%Homer Simpson\IEEEauthorrefmark{2},
	%James Kirk\IEEEauthorrefmark{3}, 
	%Montgomery Scott\IEEEauthorrefmark{3} and
	%Eldon Tyrell\IEEEauthorrefmark{4}}
	%\IEEEauthorblockA{\IEEEauthorrefmark{1}School of Electrical and Computer Engineering\\
	%Georgia Institute of Technology,
	%Atlanta, Georgia 30332--0250\\ Email: see http://www.michaelshell.org/contact.html}
	%\IEEEauthorblockA{\IEEEauthorrefmark{2}Twentieth Century Fox, Springfield, USA\\
	%Email: homer@thesimpsons.com}
	%\IEEEauthorblockA{\IEEEauthorrefmark{3}Starfleet Academy, San Francisco, California 96678-2391\\
	%Telephone: (800) 555--1212, Fax: (888) 555--1212}
	%\IEEEauthorblockA{\IEEEauthorrefmark{4}Tyrell Inc., 123 Replicant Street, Los Angeles, California 90210--4321}}
	
	
	
	
	% use for special paper notices
	%\IEEEspecialpapernotice{(Invited Paper)}
	
	
	
	
	% make the title area
	\maketitle
	
	% As a general rule, do not put math, special symbols or citations
	% in the abstract
	\begin{abstract}
	The modifiability has long been viewed as one of the important matrices of a software design with a good quality. However, there exists no obvious metrics to evaluate design modifiability. Even the understandings of modifiability are not widely agreed. This report will compare 4 different designs, and then discuss the meaning of modifiability as well as the assessments.  
	\end{abstract}
	
	% Note that keywords are not normally used for peerreview papers.
	\begin{IEEEkeywords}
	Modifiability, metrics, quantity, quality, empirical relationship.
	\end{IEEEkeywords}
	
	
	
	
	% For peer review papers, you can put extra information on the cover
	% page as needed:
	% \ifCLASSOPTIONpeerreview
	% \begin{center} \bfseries EDICS Category: 3-BBND \end{center}
	% \fi
	%
	% For peerreview papers, this IEEEtran command inserts a page break and
	% creates the second title. It will be ignored for other modes.
	\IEEEpeerreviewmaketitle
	
	
	
	\section{Introduction}
	% no \IEEEPARstart
	The modifiability of design, which means that to how much degree the design can be changed when requirements vary, is one of the important attributes when software designers evaluate the quality of their designs. To clarify the definition and effective metrics of design modifiability, this report will compare 4 designs from both quantitative and qualitative perspectives, and then by analysing the empirical relationship between them, conclude the metrics to effectively measure design modifiability.
	
	\section{Modifiablity Defination}
	Literally, the modifiability of design means the capability of the software to be modified. The modification discussed here is the changes happens on design rather than software behaviours. Under a certain change case, a valid modification means to the largest extend as possible to avoid changes on existing code, but satisfies the changing requirements \cite{2}.
	
	It is a concept widely discussed in \gls{OO} software design methodology, since, more often than not, the modifiability is reached through \gls{OO} design principles, such as abstraction, encapsulation, inheritance, polymorphism \cite{4}. It means, to some extends, the modifiability of a design can be reflected by the quantity of applying \gls{OO} design paradigm. However, we cannot simply claim that the more \gls{OO} design principles are applied \cite{3}, the more modifiability a design will has, since the \gls{OO} design principles could be misused so that have no actual effects at all. To determine the effectiveness of \gls{OO} design paradigm, the quality of design should also be concerned and tested against some change cases.
	
	In the coming sections, this report will compare the \gls{OO} design metrics of 4 designs quantitatively, and then evaluate the design quality against several change cases.
	
	\section{Quantitative Comparison}
	The modifiability can be reflected by some metrics of \gls{OO} design paradigm statistically. This results from the fact that the goal of \gls{OO} design paradigm is to better support changes than what came before. Without properly applying sufficient \gls{OO} design principles, the design will have very limited changeability.
	
	To measure the usability of the \gls{OO} design principles within a design, the \gls{NoT}, which stands for the scale of the abstraction of this design, is the first metric need to be concerned. It is hard to believe a simple design will have too much modifiability. The \gls{NoP} reflects the maintainability of semantic coherence. The \gls{NoI}, which is the abstraction of common services, can limit the communication protocols between objects. To keep the expected changes on enums at the same place, the \gls{NoE} should also be recorded here. The \gls{NoEC} will be excluded and listed separately from the \gls{NoC}, which is the direct result of applying \gls{OO} design paradigm, as the exception capturing have no impact on modifiability. 
	
	Besides these, specifically to each design, the \gls{CBO}~\cite{1}, includes all the dependencies and associations between class, should be not high if a better modifiability is expected. The \gls{DIT} and \gls{NOC}~\cite{1} are the straight reflections of the using of the inheritance principle. The \gls{WMC}~\cite{1}, which implies the count of methods per class, repeals the changeability since a class with many methods is likely to be more application specific, limiting the possibility of reusing and modifying. The static methods are effective at decoupling since they provide global accessibility. However, too much using of it will destroy the encapsulation principle, hence it is also listed \gls{NOSM} here.  
	
	In the next section, the report will firstly provide an overview of statistics of 4 designs, and then analyse them separately.
	
	\subsection{Overall}
	This section will compare the 4 design quantitatively by statistics of \gls{OO} design metrics, specifically, from the number of package, classes, exception classes, interfaces, enums and types respectively. 
	
	\begin{table}[H]
	\renewcommand{\arraystretch}{1.3}	
	\caption{The overall comparison of 4 designs.}
	\label{tab:overall}
	\centering
	\begin{tabularx}{\columnwidth}{@{}lcccc@{}}
	\toprule
	\toprule
	& \textbf{Design A} & \textbf{Design B} & \textbf{Design S1} & \textbf{Design syu702} \\ 
	\midrule
	\textbf{NoT}  & 5  & 28 & 11 & 36            \\
	\textbf{NoP}  & 1  & 6  & 4  & 7             \\
	\textbf{NoI}  & 0  & 6  & 1  & 8             \\
	\textbf{NoE}  & 1  & 2  & 2  & 0             \\
	\textbf{NoC}  & 4  & 20 & 8  & 28            \\
	\textbf{NoEC} & 0  & 0  & 0  & 0             \\	
	\bottomrule
	\bottomrule
	\end{tabularx}
\end{table}
	According to the overall statistics of each design as Table~\ref{tab:overall} shows, we can see that  either of design B and syu702 has more than 30 types totally to abstract and encapsulate different functionalities of the Kalah game. By contrast, the total \gls{NoT} in either design B or S1 is around 10, which means the using of \gls{OO} design principles within design are very limited. The smaller \gls{NoI} in design A and S1 means to lower level of common service abstraction. Meanwhile, compared with design B and syu702, the gls{NoC} in design A and S1 is insufficient. If we assume the original requirements are the same and all be met by each design, the densities of functionality implementation in each class in design A and S1 are much higher than design B and syu702. It is acknowledged that the class with high functional implementation density is hard to maintain and reuse. 
	
	\subsection{Design A}		
	\TableBlock{metrics_a}{Metrics of design A.}{
	\toprule
	\toprule
	\TabMetricHeader
	\midrule
	\TabMetricColumn{kalah}{Kalah}     {}{1}{0}{5} {1}
	\TabMetricColumn{kalah}{Player}    {}{1}{0}{2} {0}
	\TabMetricColumn{kalah}{Board}     {}{1}{0}{31}{1}
	\TabMetricColumn{kalah}{GamePlay}  {}{1}{0}{10}{0}
	\TabMetricColumn{kalah}{MoveResult}{}{0}{0}{0} {0}
	\hline
	\TabMetricColumn{}{Total}          {}{4}{0}{48}{2}
	\bottomrule
	\bottomrule
}

	As the Table~\ref{tab:overall} indicates, the design A has totally 5 types including 4 classes, 1 enum and no interface. The detailed metrics of each type are listed in Table~\ref{tab:metrics_a}. The highest \gls{CBO} in GamePlay is 2, which looks noraml. However, considering there are totally 5 types, the modification on this class should be done carefully. There exists no inheritance within this design as each class has neither parent (the maximum \gls{DIT} is 1) nor child (the maximum \gls{NOC} is 0). The high \gls{WMC} of class Board implies that this class is too heavy, having some functions which could be distributed.
	
	\subsection{Design B}
	\TableBlock{metrics_a}{Metrics of design A.}{
	\toprule
	\toprule
	\TabMetricHeader
	\midrule
	\TabMetricColumn{kalah}{Kalah}     {}{1}{0}{5} {1}
	\TabMetricColumn{kalah}{Player}    {}{1}{0}{2} {0}
	\TabMetricColumn{kalah}{Board}     {}{1}{0}{31}{1}
	\TabMetricColumn{kalah}{GamePlay}  {}{1}{0}{10}{0}
	\TabMetricColumn{kalah}{MoveResult}{}{0}{0}{0} {0}
	\hline
	\TabMetricColumn{}{Total}          {}{4}{0}{48}{2}
	\bottomrule
	\bottomrule
}
	Although there are 20 classes, 6 interfaces and 2 enums in design B according to Table~\ref{tab:overall}, it has only one effective inheritance between class Human and Player according to \gls{DIT} and \gls{NOC}. The \gls{CBO} ranges from 0 to 4, which is not too high compared with the total \gls{NoT}. It is worth noting that the \gls{NOSM} in GameFactory effectively decouple the dependence with other classes.  Generally, the methods' weight are distributed, but the \gls{WMC} of Pit and StdToplogy are still a little bit high compared with others. 
	
	\subsection{Design S1}
	\TableBlock{metrics_s1}{Metrics of design S1.}{
	\toprule
	\toprule
	\TabMetricHeader
	\midrule
	\TabMetricColumn{container}{House}     		{1}{2}{0}{1} {0}
	\TabMetricColumn{container}{Store}    		{0}{2}{0}{1} {0}
	\TabMetricColumn{container}{Container}      {0}{1}{2}{7} {0}
	\TabMetricColumn{kalah}{Board}  			{2}{1}{0}{54}{0}
	\TabMetricColumn{kalah}{Printer}			{2}{1}{0}{35}{0}
	\TabMetricColumn{kalah}{Kalah}  			{3}{1}{0}{13}{1}
	\TabMetricColumn{user}{Player}				{2}{1}{0}{9} {0}
	\TabMetricColumn{user}{User}  				{0}{0}{0}{0} {0}
	\TabMetricColumn{util}{Gam..les}  			{0}{1}{0}{0} {0}
	\TabMetricColumn{util}{Command}				{0}{0}{0}{0} {0}
	\TabMetricColumn{util}{Order}  				{0}{0}{0}{0} {0}
	\bottomrule
	\bottomrule
	%\TabMetricFooter
}

	For design S1, there are 8 classes, 1 interface and and 2 enums recorded on Table~\ref{tab:overall}. Despite House and Store are sub-classes of Container, the inheritance is not effective here since on Table~\ref{tab:metrics_s1}, the \gls{WMC} of them are low. This results in 2 "god classes", Board and Printer, which do complicated tasks that should have been distributed to other classes properly. What's worse, the classes with higher \gls{WMC} also have more \gls{CBO}.
	
	\subsection{Design syu702}
	\TableBlock{metrics_syu702}{Metrics of design syu702.}{
	\toprule
	\toprule
	\TabMetricHeader
	\midrule
	\TabMetricColumn{kalah}{Kalah}     				{0}{4}{0}{9} {1}
	\TabMetricColumn{kalah}{Game}      				{2}{3}{1}{11}{0}
	\TabMetricColumn{.structure}{Group}     		{0}{4}{0}{4} {0}
	\TabMetricColumn{.structure}{PlayBoard}  		{1}{3}{0}{5} {0}
	\TabMetricColumn{.structure}{House}				{0}{2}{0}{3} {0}
	\TabMetricColumn{.structure}{Store}  			{0}{2}{0}{2} {0}
	\TabMetricColumn{.structure}{Container}			{2}{1}{2}{4} {0}
	\TabMetricColumn{.pla..ion}{KalahPlayer}     	{1}{3}{0}{6} {0}
	\TabMetricColumn{.pla..ion}{SowAction}  		{1}{3}{0}{10}{0}
	\TabMetricColumn{.pla..ion}{GamePlayer}			{0}{2}{1}{5} {0}
	\TabMetricColumn{.pla..ion}{GameAction}  		{0}{2}{1}{3}{0}
	\TabMetricColumn{.pla..ion}{Gam..ent}			{1}{1}{2}{3} {0}
	\TabMetricColumn{.pla..ion}{ActionPool}			{1}{1}{0}{5} {0}
	\TabMetricColumn{.pla..ion}{IPerson}  			{1}{0}{0}{0} {0}
	\TabMetricColumn{.pla..ion}{IAction}			{0}{0}{0}{0} {0}
	\TabMetricColumn{.selector}{Con..tor}     		{0}{3}{1}{9} {0}
	\TabMetricColumn{.selector}{Cir..tor}  			{2}{2}{2}{2} {0}
	\TabMetricColumn{.selector}{Lin..tor}			{1}{2}{1}{3} {0}
	\TabMetricColumn{.selector}{Selector}  			{0}{1}{2}{3} {0}
	\TabMetricColumn{.selector}{ISe..ble}			{0}{0}{0}{0} {0}
	\TabMetricColumn{.wrapper}{SeedNum}  			{0}{2}{0}{3} {0}
	\TabMetricColumn{.wrapper}{Wrapper}				{0}{1}{1}{4} {0}
	\TabMetricColumn{.wrapper}{IGetter}  			{0}{0}{0}{0} {0}
	\TabMetricColumn{.wrapper}{ISetter}				{0}{0}{0}{0} {0}
	\TabMetricColumn{.dump}{DumpVisitor}  			{0}{1}{0}{9} {0}
	\TabMetricColumn{.dump}{Dum..ton}				{1}{1}{0}{2} {1}
	\TabMetricColumn{.dump}{IArgVisitor}  			{2}{0}{0}{0} {0}
	\TabMetricColumn{.dump}{IDumpable}				{1}{0}{0}{0} {0}
	\TabMetricColumn{.rule}{GameOver}  				{1}{1}{0}{2} {0}
	\TabMetricColumn{.rule}{Rul..ton}				{1}{1}{0}{2} {1}
	\TabMetricColumn{.rule}{EmptyHouse}  			{1}{1}{0}{2} {0}
	\TabMetricColumn{.rule}{HoldTurn}				{1}{1}{0}{2} {0}
	\TabMetricColumn{.rule}{RuleObserver}  			{1}{1}{0}{4} {0}
	\TabMetricColumn{.rule}{ChangeTurn}				{1}{1}{0}{2} {0}
	\TabMetricColumn{.rule}{Cap..ent}  				{1}{1}{0}{3} {0}
	\TabMetricColumn{.rule}{IRule}					{0}{0}{0}{0} {0}
	\bottomrule
	\bottomrule
	%\TabMetricFooter
	\TabFooter{1}{The package name which starts with "." means it is a sub-package of "kalah", for example, .player=kalah.player.}
	\TabFooter{2}{".." is used to reduce the length of the type name.}
}

	The detailed metric statistics about design syu702, which has 28 classes and 8 interfaces on Table~\ref{tab:overall}, are listed on Table~\ref{tab:metrics_syu702}. The low \gls{CBO} means to good code modularity and encapsulation.  In this design, the inheritance is effectively used more than one time, which reduce the code duplication. The \gls{WMC} of each class ranges from 2 to 11, which means there is no "god class". Even completely rewriting some classes will not affect a lot implementation due to the low \gls{WMC}.
	
	\subsection{summary}
	From the statistical result of the \gls{OO} design metrics, design B and design syu702 have more modifiability then other 2. To be specific, the design syu702 is best among them.  
	 
	\section{Qualitative Comparison}
	To be complementary, the quality of design, which involves the adaptations to change cases and some design patterns, should also be discussed, as sometimes the \gls{OO} design principles could be inappropriately applied. To check the real effects of applying the \gls{OO} design paradigm, the 4 designs with their UML diagram will be evaluated against the change cases listed in Table~\ref{tab:cases}.
	
	\begin{table}[H]
	\renewcommand{\arraystretch}{1.3}	
	\caption{Changes cases to be considered during evaluation.}
	\label{tab:cases}
	\centering
	\begin{tabularx}{\columnwidth}{@{}lX@{}}
		\toprule
		\toprule
		\textbf{Case} & \textbf{Description} \\
		\midrule
		\textbf{DIR} & Change the direction of sowing seeds \\
		\textbf{UND} & Provide an Undo move \\
		\textbf{S/L} & Provide Save/Load for partially completed games \\
		\textbf{EPY} & Change the capture rule to, a capture can take place even when the opposite house is empty \\
		\textbf{TKC} & Change the capture rule to, a capture takes place if the last seed falls into an opponent's house and that house has an even number of seeds, then all seeds in that house are captured \\
		\textbf{RBT} & Allow the possibility that the second player is played by the computer, that is, is a robot \\
		\textbf{PRE} & Change in presentation of playing board, for example, using "*" for each seed \\
		\bottomrule
		\bottomrule
	\end{tabularx}
\end{table}

	\subsection{Design A}
	The UML diagram of design A is demonstrated at appendix A.
	
	Considering there are very little abstraction of objects and solid encapsulation, but neither inheritance nor polymorphism to enable modification in design A, the modifiability is very limited. It means no matter what kind of modification the developer want to bring to any class in design A, he has to not only edit the target class, but also modify other classes associated. None of the change cases will be easily met in this case.
	
	\subsection{Design B}
	The UML diagram of design B is demonstrated at appendix B.
	
	The interface PitTopology makes changing the direction of sowing not difficult. The package kalah.move brings more scalability to adding a new move  such as "UND" to the game. There is a design pattern named "simple factory" applied in GameFactory. With the class Game, GameState and GameFactory, the "S/L" change case can be easily implemented by saving or loading a instance of GameState in GameFactory. To change the capture rule to "EPY" or "TKC", the interface BoardStateProcessor and BoardStateProperty should be realizated like KalahStdCaptureProcessor and KalahStdCaptureProperty, but with different logics. The abstract class Player can be extended to create a new robot player, just similar with the class Player. The only problem of this design regarding to these change cases is to change the IO output. If any new output, such as "*" to represent seeds on playing board, is required, the developers have to seek through all the classes outputting something to the IO, and change the code by hands.
	
	\subsection{Design S1}
	The UML diagram of design S1 is demonstrated at appendix C.
	
	The design S1 has a typical tree structure. A kalah game has a Board and a Printer, a Board has several Players, each Player holds some Houses and one Store. Although there exists a inheritance, the only one, between Container and House and Store, which means it is easy to add a new kind of place to store and sow the seeds, this inheritance is not effective as it brings no changeability to the tree structure we mentioned before. The interface User makes adding a new user, such as a robot, possible, but the actual player actions are required to be implemented in class Board. The enums Command and GameVariables bring more ease on adding new moves such as "UND", but this changeability is behaviour based rather than design based, which means the associated code is still expected in class Kalah. Therefore, actually this design is just a little bit better than design A on modifiability, even some \gls{OO} design principles are used. Despite more classes and enums are used in this design compared with design A, which means more design abstraction and object categorization, the inappropriate associations between classes, such as "a player has some houses and a store" and "a board has players", reduce the quality of design. The class Printer concentrates on IO displaying, which narrowed the range of code to change under "PRE". However, the classes which want to print has to call the instance of Printer somehow. This leads to high potential \gls{CBO} on class printer if many classes have displaying requirements.
	
	\subsection{Design syu702}	
	The UML diagram of design syu702 is demonstrated at appendix D.
		
	The design of package kalah.selector is similar as a design pattern named "iterator", but not the typical one. As the name implies, the typical iterator provides a structure by which the developers can easily iterate elements in a collection. Besides the iteration function which is the same as a iterator, a selector focuses on returning the expected elements. With this design, the change case "DIR" can be easily implemented by creating a new selector named "ReversingSelector" and then letting other classes who expect this functionality inherit from it.
	
	The design of package kalah.rule makes the kalah game more scalable. Here a design pattern named "observer" is applied to decouple the rules and their effects on the kalah game. With this design, the implementations of change cases "EPY" and "TKC", or even other more new rules, are as simple as to create a new class to realizate the IRule interface and implement how to affect the game in the method Affect. The instance of this class should be properly observed by RuleObserver at where it occurs after the player's certain action, for example, the SowAction.
	
	The design of package kalah.playeraction aims to define game players and their different actions. The class ActionPool can load different actions and index them by literals. With the help of ActionPool, the change case UND can be achieved by creating a new class which inherits from GameAction. The instance of this class should be registered at th beginning of this game in ActionPool and acted by game players when it is required. The robot player change case "RBT" can be implemented by adding a new class inherited from GamePlayer and overriding the method named Act.
	
	The design patterns "visitor" and "singleton" are used in package kalah.dump. The class DumpSingleton makes DumpVisitor globally accessible. The class DumpVisitor wraps the IO and centralizes all the dumping functions. Any class who wants to display into IO can easily inherits from interface IDumpable and puts its dumping function in DumpVisitor. With the design in kalah.dump package, the change case PRE is easy to be matched by implementing interface IDumpable and add new dumping function in DumpVisitor.
	
	The "S/L" change case can be met by adding a serializer which enable to serialize and de-serialize the instance of Game to a local file. However, honestly speaking it is out of the original design.
	
	\subsection{Summary}
	Based on the the analysis of each design against the change cases, design B and syu702 can satisfy the same number of changes cases. As regarding to design A and S1, design S1 is a little bit better than A since the designer starts to utilize some \gls{OO} design methods and put associated code together, however, the applying is sometimes not effective from the result.
	
	\section{Empirical Relationship}
	From the previous sections, we figure out that the design A, which has the fewest number of classes, has the fewest modifiability so that no change cases can be easily satisfied. The S1 have larger quantity of classes and interfaces, however, they are not effective in decoupling and functionality distributing. The modifiability of design S1 is close to design A, but higher since the awareness of using interface and enums. Both of design B and syu702 have sufficient quantity of abstraction and inheritance applied and can satisfy the equal number of the change cases with limited modifications. However, considering the outstanding performance the design syu702 has played in quantitative comparison, the design syu702 wins design B slightly on more modifiability.
	Based on the analysis previous, if the M(x) stands for the modifiability of design x, the relationship of modifiability between each design is as Equation~\ref{eq:realtionship} implies:
	
	\begin{equation} 
	M(syu702) > M(B) > M(S1) > M(A)
	\label{eq:realtionship} 
	\end{equation}
	
	
	
	% An example of a floating figure using the graphicx package.
	% Note that \label must occur AFTER (or within) \caption.
	% For figures, \caption should occur after the \includegraphics.
	% Note that IEEEtran v1.7 and later has special internal code that
	% is designed to preserve the operation of \label within \caption
	% even when the captionsoff option is in effect. However, because
	% of issues like this, it may be the safest practice to put all your
	% \label just after \caption rather than within \caption{}.
	%
	% Reminder: the "draftcls" or "draftclsnofoot", not "draft", class
	% option should be used if it is desired that the figures are to be
	% displayed while in draft mode.
	%
	%\begin{figure}[!t]
	%\centering
	%\includegraphics[width=2.5in]{myfigure}
	% where an .eps filename suffix will be assumed under latex, 
	% and a .pdf suffix will be assumed for pdflatex; or what has been declared
	% via \DeclareGraphicsExtensions.
	%\caption{Simulation results for the network.}
	%\label{fig_sim}
	%\end{figure}
	
	% Note that the IEEE typically puts floats only at the top, even when this
	% results in a large percentage of a column being occupied by floats.
	
	
	% An example of a double column floating figure using two subfigures.
	% (The subfig.sty package must be loaded for this to work.)
	% The subfigure \label commands are set within each subfloat command,
	% and the \label for the overall figure must come after \caption.
	% \hfil is used as a separator to get equal spacing.
	% Watch out that the combined width of all the subfigures on a 
	% line do not exceed the text width or a line break will occur.
	%
	%\begin{figure*}[!t]
	%\centering
	%\subfloat[Case I]{\includegraphics[width=2.5in]{box}%
	%\label{fig_first_case}}
	%\hfil
	%\subfloat[Case II]{\includegraphics[width=2.5in]{box}%
	%\label{fig_second_case}}
	%\caption{Simulation results for the network.}
	%\label{fig_sim}
	%\end{figure*}
	%
	% Note that often IEEE papers with subfigures do not employ subfigure
	% captions (using the optional argument to \subfloat[]), but instead will
	% reference/describe all of them (a), (b), etc., within the main caption.
	% Be aware that for subfig.sty to generate the (a), (b), etc., subfigure
	% labels, the optional argument to \subfloat must be present. If a
	% subcaption is not desired, just leave its contents blank,
	% e.g., \subfloat[].
	
	
	% An example of a floating table. Note that, for IEEE style tables, the
	% \caption command should come BEFORE the table and, given that table
	% captions serve much like titles, are usually capitalized except for words
	% such as a, an, and, as, at, but, by, for, in, nor, of, on, or, the, to
	% and up, which are usually not capitalized unless they are the first or
	% last word of the caption. Table text will default to \footnotesize as
	% the IEEE normally uses this smaller font for tables.
	% The \label must come after \caption as always.
	%
	%\begin{table}[!t]
	%% increase table row spacing, adjust to taste
	%\renewcommand{\arraystretch}{1.3}
	% if using array.sty, it might be a good idea to tweak the value of
	% \extrarowheight as needed to properly center the text within the cells
	%\caption{An Example of a Table}
	%\label{table_example}
	%\centering
	%% Some packages, such as MDW tools, offer better commands for making tables
	%% than the plain LaTeX2e tabular which is used here.
	%\begin{tabular}{|c||c|}
	%\hline
	%One & Two\\
	%\hline
	%Three & Four\\
	%\hline
	%\end{tabular}
	%\end{table}
	
	
	% Note that the IEEE does not put floats in the very first column
	% - or typically anywhere on the first page for that matter. Also,
	% in-text middle ("here") positioning is typically not used, but it
	% is allowed and encouraged for Computer Society conferences (but
	% not Computer Society journals). Most IEEE journals/conferences use
	% top floats exclusively. 
	% Note that, LaTeX2e, unlike IEEE journals/conferences, places
	% footnotes above bottom floats. This can be corrected via the
	% \fnbelowfloat command of the stfloats package.
	
	
	
	
	\section{Conclusion}
	From the comparison of the 4 designs, we can learn that the modifiability is such a abstract concept that varies on different environments. Even though it can be measured by some \gls{OO} design metrics quantitatively like what are demonstrated in quantitative comparison section, the real effects are still required to be tested against specific change cases because there is no guarantee between the statistic of metrics and the actual outcomes.  
	
	
	
	
	% conference papers do not normally have an appendix
	\appendices 
	\clearpage
	\section{UML Diagram of Design A}
	\ImageBlock{A}{The architecture of design A.}
	\clearpage
	\section{UML Diagram of Design B}
	\ImagePage{B}{The architecture of design B.}
	\clearpage
	\section{UML Diagram of Design S1}
	\ImagePage{S1}{The architecture of design S1.}
	\clearpage
	\section{UML Diagram of Design syu702}
	\ImagePage{syu702}{The architecture of design syu702.}
	\clearpage

	% use section* for acknowledgment
	\section*{Acknowledgment}
	
	
	I would like to express my special thanks of gratitude to my lecturer Ewan, whose critical thinking skills and humour in lecturing really impress me and teach me a lot. If there exists any marker on this assignment, I think he/she definitely also deserves my thanks, because reading such a long report with tables and diagrams must be torturing and time-consuming.
	
	
	\printglossaries
	
	
	% trigger a \newpage just before the given reference
	% number - used to balance the columns on the last page
	% adjust value as needed - may need to be readjusted if
	% the document is modified later
	%\IEEEtriggeratref{8}
	% The "triggered" command can be changed if desired:
	%\IEEEtriggercmd{\enlargethispage{-5in}}
	
	% references section
	
	% can use a bibliography generated by BibTeX as a .bbl file
	% BibTeX documentation can be easily obtained at:
	% http://mirror.ctan.org/biblio/bibtex/contrib/doc/
	% The IEEEtran BibTeX style support page is at:
	% http://www.michaelshell.org/tex/ieeetran/bibtex/
	\bibliographystyle{IEEEtran}
	% argument is your BibTeX string definitions and bibliography database(s)
	\bibliography{IEEEabrv,references}
	%
	% <OR> manually copy in the resultant .bbl file
	% set second argument of \begin to the number of references
	% (used to reserve space for the reference number labels box)
	
	
	
	
	% that's all folks
\end{document}


